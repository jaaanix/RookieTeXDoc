\chapter{Was ist Latex}
 \section{Der Ursprung}
 Die Software TeX ist ein von Donald E. Knuth entwickletes System um Texte jeglicher Art, aber vor allem Wissenschaftliche Texte zu verfassen. Aufbauend auf der Basis-Software TeX entwickelt Leslie Lamport seit den 1980er Jahren eine Sammlung von TeX-Makros (genannt LaTeX). Diese Tex-Makros dienen hauptsächlich dazu, dass verfassen von Texten mit TeX zu vereinfachen.

 \section{WYSIWYG}
 WYSIWYG steht für What You See Is What You Get, dieses Prinzip beschreibt ein Verhalten von Textverarbeitungsprogrammen, in denen man schon während des Verfassens die finale Formatierung und das Aussehen des Textes zu sehen bekommt. Das bekannteste Beispiel für eine Software, welche nach diesem Prinzip funktioniert ist das Textverarbeitungsprogramm Microsoft Word.

 TeX hingegen verzichtet auf diesen Ansatz. Genau wie bei der Auszeichnungssprache (markup language) HTML wird die Formatierung, Gliederung und das Aussehen des Produkts durch eine eigene Syntax abgebildet und ist mit dem Inhalt vermischt. Um das fertige Produkt, das Dokument zu erstellen, ist ein Kompiliervorgang nötig. Dieses Prinzip wird auch als WYSIWYAF (What You See Is What You Asked For) bezeichnet.

 \section{Haupteinsatzgebiete für TeX}
 Besonders im naturwissenschaftlichen Bereich bietet sich die Auszeichnungssprache TeX an, da TeX eine komfortable Möglichkeit bietet Formeln oder Programmcode darzustellen. Ebenso lassen sich Abstände genau definieren (in Millimeter, Centimeter usw.), was ein häufig wichtiges Kriterium für Wissenschaftliche Arbeiten darstellt. TeX bietet sich für Bachelorarbeiten, Masterarbeiten, Dissertationen und andere Arbeiten welche mathematische Inhalte vermittelnan an.

 Weitere komfortable und professionelle Möglichkeiten bietet TeX unter anderem zum erstellen von Inhaltsverzeichnisses, Abbildungsverzeichnissen, Tabellenverzeichnissen, Literaturverzeichnissen und Glossaren.

\chapter{Was zeichnet Atom aus}
 \section{Atom der hackable Text Editor}
 Atom ist ein vom Online-Dienst GitHub seit 2014 verfügbarer Text-Editor. Er ist frei verfügbar, open-source (MIT License) und plattformunabhängig. Entwicklelt wurde der Text-Editor mit CoffeeScript, JavaScript, Less und HTML. Der Entwickler GitHub wirbt mit dem Slogan "`A hackable text editor for the 21st Century"'. Damit wird sich vor allem auf die Anpassbarkeit des Aussehens und der Verhaltens des Editor bezogen.

 Ebenso wie der Text-Editor selbst, sind auch die in Atom verfügbaren Packages (Plugins) in CoffeeScript bzw. JavaScript programmiert. Die Packages erweitern den Editor in seiner Funktionalität und werden sowohl von GitHub selber, so wie auch durch die Community erstellt und gepflegt.

\chapter{Das Atom Package RookieTeX}
\section{Nutzen und Einsatzgebiet des Packages}
