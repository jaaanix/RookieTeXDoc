\chapter{Was ist Latex}
 \section{Der Ursprung}
 Die Software TeX ist ein von Donald E. Knuth entwickletes System um Texte jeglicher Art, aber vor allem Wissenschaftliche Texte zu verfassen. Aufbauend auf der Basis-Software TeX entwickelt Leslie Lamport seit den 1980er Jahren eine Sammlung von TeX-Makros (genannt LaTeX). Diese Tex-Makros dienen hauptsächlich dazu, dass verfassen von Texten mit TeX zu vereinfachen.

 \section{WYSIWYG}
 WYSIWYG steht für What You See Is What You Get, dieses Prinzip beschreibt ein Verhalten von Textverarbeitungsprogrammen, in denen man schon während des Verfassens die finale Formatierung und das Aussehen des Textes zu sehen bekommt. Das bekannteste Beispiel für eine Software, welche nach diesem Prinzip funktioniert ist das Textverarbeitungsprogramm Microsoft Word.

 LaTeX hingegen verzichtet auf diesen Ansatz. Genau wie bei der Auszeichnungssprache (markup language) HTML wird die Formatierung, Gliederung und das Aussehen des Produkts durch eine eigene Syntax abgebildet und ist mit dem Inhalt vermischt. Um das fertige Produkt, das Dokument zu erstellen, ist ein Kompiliervorgang nötig. Dieses Prinzip wird auch als WYSIWYAF (What You See Is What You Asked For) bezeichnet.

\section{Einsatzgebiete für LaTeX}
- Bachelorarbeiter, Masterarbeiter, Dissertationen
- Mathematik und Naturwissenschaften
- komfortable Möglichkeiten zum erstelen von Formeln
