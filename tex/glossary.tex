\newglossaryentry{package}{name={Package}, description={Im Kontext dieser Dokumentation beschreibt das Package eine Erweiterung (auch Plugin gennant) welche für den Text Editor Atom von GitHub entwickelt werden kann, um dessen Funktionen zu erweitern}}

\newglossaryentry{scope}{name={Scope},
description={Die im Atom Editor verwendeten Scopes, beschreiben den Typ bzw. das Format einer im Editor geöffneten Datei, in welchem Funktionen eines Packages genutzt werden können}}

\newglossaryentry{library}{name={Library},
description={Libraries fassen Klassen und Funktionen einer Programmiersprache zusammen und können modular in verschiedenen Softwareprojekten wiederverwendet werden}}

\newglossaryentry{regex}{name={Regex},
description={Regular Expressions (deutsch: reguläre Ausdrücke) dienen dazu Muster in Zeichenketten zu erkennen und zu filtern}}

\newglossaryentry{plaintext}{name={Plain Text},
description={Plaint Text ist unformatierter Text, welcher lediglich Informationen über die Kodierung, also z.B. UTF8 enthält}}

\newglossaryentry{commandpalette}{name={Command Palette},
description={Die Command Palette ist ein Eingabefeld, welches schon während der Eingabe Vorschläge der verfügbaren Befehle des Editors und dessen Packages liefert. Dieses Eingabefeld im Editor zu jeder Zeit via Tastenkombination aufgerufen werden}}
